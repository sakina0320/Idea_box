\documentclass[12pt]{article}
\usepackage[margin=2.54cm]{geometry}
\usepackage{enumitem}
\usepackage{amsfonts}
\usepackage{amsmath}
\usepackage{amssymb}
\usepackage{tikz}
\usepackage{listings}
\usepackage{graphicx}
\usepackage{float} % This lets you use [H] for figure.
\usepackage{subcaption}
\usepackage[final]{pdfpages}
\usepackage{multicol}
\usepackage{multirow}
\usepackage{setspace}
\doublespacing
\usepackage{natbib}
\usepackage{hyperref}
\usepackage{mathtools}
\DeclarePairedDelimiter\Floor\lfloor\rfloor
\DeclarePairedDelimiter\Ceil\lceil\rceil

\begin{document}
\textbf{\underline{A List of Research Ideas}} \\
Q $\rightarrow$ Research questions; H $\rightarrow$ Hypothesis; C $\rightarrow$ Context; D: $\rightarrow$ Data requirement or data to be used; T $\rightarrow$ Thinking behind it; E $\rightarrow$ How to deal with endogeneity
\begin{enumerate}
	\item \textbf{The relationship between reproductive decisions and land quality}
		\begin{itemize}
			\item[Q:] Do land characteristics affect the number of children a HH have?
			\item[H:] The more labor intensive the HH's plots are, the more children the HH has.
			\item[C:] Rural, developing countries where most people are subsistence farmers 
			\item[T:] There are many evidences that show strong correlations between income and education, and fertility. It is also often said that poor, uneducated farmers tend to have a large number of children, because they need hands to work on their farms. If this story is true, there can be a variation in the number of children caused by factors that make agricultural production more labor intensive. Such factors can be soil characteristics that require much work to make it productive, land location (hilly plots require more work) and so forth.
		\end{itemize}
		
	\item \textbf{Change of the status quo and privileges in Mali}
		\begin{itemize}
			\item[Q:] Do the Bambaras and Dogons have a higher propensity to get public sector jobs compared to the Parlums because they are wealthier with remittances?
			\item[H:] The Bambaras and Dogons are more likely to get public sector job after controlling for education and income than the Parlums.
			\item[C:] Mali
			\item[D:] EMOP. The questionnaire records in what language an interview has been conducted, although this is not included in the data set. Can we have access to the information?
			\item[T:] Begin wealthy seems to influence how favorably one view another person. Are the Bambaras and Dogons viewed more favorably now that they are wealthier than in the past?
			\item[E:] I have to untangle remittances, education and income.
		\end{itemize}
	
	\item \textbf{Workers' effort level}
		\begin{itemize}
			\item[Q:] Does the lack of long term career opportunities deter people from inserting high effort level? (Explore other reasons)
			\item[C:] The Appalachia regions
			\item[T:] Many business owners here complain how hard it is to find reliable workers here. I suspect perhaps that workers don't see a reason to insert effort (for example showing up at work consistently) because they know most of the jobs available here are temporary or part-time.
		\end{itemize}	
	
	\item \textbf{Women's time poverty and social networks}
		\begin{itemize}
			\item[Q:] Do women with lower-quality social networks receive less help from other women?
			\item[H:] Women who are more isolated in her social networks are less likely to get help on child rearing, household chores, and other care works.
			\item[C:] Rural, developing countries
			\item[T:] Women in rural, agricultural settings often help each other with child care and other tasks. I want to understand how they help each other, and if there is gain in time due to helping each other, how do they use gained time.
		\end{itemize}

	\item \textbf{The effect of better roads on social networks in former conflict zones}
		\begin{itemize}
			\item[Q:] Does increased interaction due to the introduction of better roads aggravate or improve group (ethnic, class etc) divisions?
			\item[D:] Public transport data (possibly experimental), and social networks data.
		\end{itemize}	

	\item \textbf{Quantifying the benefits of weddings}
		\begin{itemize}
			\item[Q:] How much do newly weds (and their families) gain from having weddings? 
			\item[H:] Aside from gains in in-kind and monetary gifts, newly-weds and their families earns in social capital. 
			\item[D:] Wedding spending, economics and social status, income and wealth levels, how much gifts received, social capital (How do we measure gains in social capital?) 
		\end{itemize}
	
	\item \textbf{Music and risk behavior}
		\begin{itemize}
			\item[Q:] Does music have impact on risk behavior?
			\item[D:] Lab experiment 
		\end{itemize}
	
	\item \textbf{Gender identity and economic choices}
		\begin{itemize}
			\item Do some lab experiment?
		\end{itemize}

%	\item \textbf{Employer's expectation of the future productivity of female employees accurate?}
%		\begin{itemize}
%			\item[Q:] 
%			\item[H:]
%			\item[C:] 
%			\item[T:] 
%			\item[E:] 
%		\end{itemize}

%	\item Mobile phone introduction and social network
%		\begin{itemize}
%			\item[Q:] 
%			\item[H:]
%			\item[C:]
%			\item[D:]
%			\item[T:] 
%			\item[E:] 
%		\end{itemize}

%	\item Think about the whole yoshi system. Why does it exist?
%		\begin{itemize}
%			\item[Q:] 
%			\item[H:]
%			\item[C:] 
%			\item[T:] 
%			\item[E:] 
%		\end{itemize}	

%	\item Empathy and economics
%		\begin{itemize}
%			\item[Q:] 
%			\item[H:]
%			\item[C:] 
%			\item[T:] 
%			\item[E:] 
%		\end{itemize}

%	\item Why do development projects affect men and women differently?
%		\begin{itemize}
%			\item[Q:] 
%			\item[H:]
%			\item[C:] 
%			\item[T:] 
%			\item[E:] 
%		\end{itemize}
	
%	\item Trust and coordination in agricultural production
%		\begin{itemize}
%			\item[Q:] 
%			\item[H:]
%			\item[C:] 
%			\item[T:] 
%			\item[E:] 
%		\end{itemize}
\end{enumerate}


\end{document}